\chapter{Sistema actual}
Actualmente, el desarrollo profesional de juegos es una tarea muy compleja, ya que el uso de los �ltimos modelos de consolas de videojuegos significa una gran dificultad. Por ejemplo, las consolas PlayStation 3 y XBox 360 llevan en su interior un procesador Cell de 8 n�cleos de arquitectura RISC y dise�o SIMD. Esto significa que estas consolas pueden ejecutar simult�neamente 8 procesos del mismo programa y trabajar cada uno de estos procesos con multitud de datos simult�neamente. Esto significa, que para utilizar por completo la potencia de estas consolas es necesario un dise�o basado en partes separadas en procesos, con el fin de mantener ocupado lo m�s posible el procesador.

As�, hoy d�a es muy com�n dedicar un hilo a cada parte del programa que sea divisible. T�picamente, se dedica un hilo al procesamiento de la entrada, otro al de la l�gica del juego, otro a los gr�ficos, otro al sonido, otro a la red, etc. Esto significa que el juego debe ser dividido en partes separadas, y que puedan ser sincronizadas con la mayor facilidad y velocidad posibles. Esto ha favorecido la aparici�n de ingenios o motores dedicados a atender cada una de estas necesidades por separado, y que luego se integran en el desarrollo del juego.
