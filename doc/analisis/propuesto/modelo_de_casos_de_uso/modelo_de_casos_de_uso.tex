\chapter{Modelo de casos de uso}
En este cap�tulo se describir�n los casos de uso principales.

\section{Descripciones generales de actores}
\begin{itemize}
\item \textbf{Desarrollador} : El desarrollador utiliza el sistema para desarrollar su videojuego.
\item \textbf{Usuario} : El usuario en un caso de uso suele ser el propio software a�adido por el desarrollador para utilizar el framework. A menudo tambi�n ser� cualquier otra parte del sistema que dependa de la que se est� describiendo actualmente. En esos casos en la descripci�n del caso de uso se indicar� qui�n es un buen candidato a usar el sistema.
\item \textbf{Main Loop} : A menudo, el bucle principal de la aplicaci�n iniciar� un caso de uso.
\item \textbf{Proceso} : Muchas veces, parte de la l�gica del juego, expresada como procesos, iniciar� un caso de uso.
\end{itemize}
\section{Diagrama de modelo de casos de uso}

\section{General: bucle principal}
El modelo de ejecuci�n general del programa. Est� basado en el modelo de ejecuci�n de un programa interactivo, adaptado para este sistema.
\subsection{Flujo b�sico}
\begin{enumerate}
\item El sistema se inicializa.
\item Mientras deba continuar la ejecuci�n del sistema:
	\begin{enumerate}
	\item El jugador genera eventos de entrada si desea.
	\item Si hay eventos de entrada:
		\begin{enumerate}
		\item El sistema procesa los eventos de entrada.
		\end{enumerate}
	\item El sistema procesa la informaci�n del juego.
	\item El sistema muestra el resultado del procesamiento.
	\item El jugador ve el resultado de sus acciones.
	\end{enumerate}
\item El sistema libera recursos y finaliza su ejecuci�n.
\end{enumerate}
\subsection{Requisitos especiales}
\subsection{Precondiciones}
\subsection{Postcondiciones}
\subsection{Puntos de extensi�n}
\subsection{Relaciones}
\subsection{Otros diagramas}


\section{Objetos: emparentar objeto}
Cambia de padre a un objeto. Sirve para desplazar ese objeto por el �rbol de objetos.
\subsection{Flujo b�sico}
\begin{enumerate}
\item El usuario ordena al objeto cambiar de padre, y especifica un nuevo padre.
\item Si el objeto ten�a otro padre anteriormente:
	\begin{enumerate}
	\item El objeto se libera de la dependencia con su padre anterior.
	\end{enumerate}
\item El objeto establece una dependencia con su nuevo padre.
\end{enumerate}

\subsection{Requisitos especiales}
\subsection{Precondiciones}
\subsection{Postcondiciones}
\subsection{Puntos de extensi�n}
\subsection{Relaciones}
\subsection{Otros diagramas}
