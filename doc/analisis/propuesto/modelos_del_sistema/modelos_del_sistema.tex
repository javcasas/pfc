\newcommand{\flecha}[0]{-\textgreater\ }

\chapter{Modelos del sistema}
\section{Sistema gr�fico}
Se propone dividir el motor gr�fico en varios niveles, con el fin de simplificar las relaciones y reducir la dependencia. Por ello se proponen tres niveles:
\begin{itemize}
\item \textbf{Nivel de geometr�a} : Trata con v�rtices, pol�gonos y mallas
\item \textbf{Nivel de objetos} : Trata con los objetos que componen una escena, c�maras y luces
\item \textbf{Nivel de dibujado} : Trata de c�mo manejar las escenas que componen la imagen final que se muestra en pantalla
\end{itemize}



\subsection{Nivel de geometr�a}
Este nivel organiza el almacenamiento y proceso de las mallas de pol�gonos y los elementos que las componen. Est� organizado como clases de almacenamiento de datos (clases entidad) ya que apenas hay procesamiento en este subsistema, al menos respecto a la parte que ser� implementada en esta revisi�n.

\subsection{Nivel de objetos}
Este nivel organiza el procesamiento de las entidades que componen una escena, trat�ndolas como entidades indivisibles. Incluye el caso de uso \emph{emparentar}, que se encarga de describir c�mo unos objetos pueden depender de otros geom�tricamente hablando.

\subsection{Nivel de dibujado}
Este nivel se ha organizado para poder controlar varias pantallas con distintos sistemas e im�genes en cada una.

\section{Motor de procesos}
Este subsistema se encarga de proporcionar un esqueleto para la l�gica del juego. Para ello propone dividir dicha l�gica en peque�os elementos que se ejecutan concurrentemente como los procesos en un sistema operativo. Incluye los casos de uso \emph{ejecutar\_procesos}, que se encarga de dar tiempo de CPU a cada proceso; y \emph{cambiar\_prioridad}, que se encarga de decidir el orden en que se ejecutar�n los procesos.
