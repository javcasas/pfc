\chapter{Visi�n general}
Como ya se ha comentado, en este sistema pretendemos conseguir un mecanismo de aceleraci�n del desarrollo de un videojuego, ofreciendo para ello una parte del trabajo ya hecho. As� esa parte tan s�lo hay que retocarla y adaptarla al caso concreto. Por otra parte debe ser sencillo de manejar, con el fin de evitar que el coste de entrenar a los ingenieros para usarlo sea menor que el de desarrollar uno nuevo desde cero.

Por ello se busca que el sistema sea sencillo y extendible. Esto permitir� adaptarlo a los nuevos tiempos y a nuevas tecnolog�as. Esto implica que el sistema estar� desacoplado de bibliotecas concretas, con el fin de evitar que la desaparici�n de estas bibliotecas signifique el fin del sistema, y con la idea de poder adaptarlo a  bibliotecas nuevas.

Es evidente por lo indicado hasta ahora que este sistema est� orientado a desarrolladores.

En esta documentaci�n se distinguir� a dos tipos de usuarios:
\begin{itemize}
\item \textbf{El desarrollador} : Se le nombrar� normalmente como el usuario del producto, y es quien utiliza el sistema para desarrollar un videojuego.
\item \textbf{El jugador} : Es el usuario final, el que utiliza el videojuego desarrollado con el sistema que aqu� se expone.
\end{itemize}
