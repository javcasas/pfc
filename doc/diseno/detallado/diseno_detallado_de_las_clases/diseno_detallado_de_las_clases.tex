\chapter{Dise�o detallado de las clases}

\section{Cargador\_malla\_simple}
Se encarga de cargar e instanciar Malla\_simple. Incluye un lexer y un parser que se encargan de la lectura del fichero. Tanto el lexer como el parser se har�n con herramientas ya disponibles: OcamlLex y OcamlYACC.

OcamlLex y OcamlYACC generan cada uno un m�dulo nuevo que es utilizado por Cargador\_malla\_simple. Estos m�dulos se llaman \emph{Cargador\_malla\_lexer} y \emph{Cargador\_malla\_parser}.

\begin{itemize}
\item \textbf{Relaciones:}
	\begin{itemize}
	\item <<instantiate>> Clases del tipo *\_simple. \newline
		Instancia Malla\_simple y las clases que utilice Malla\_simple. Adem�s, realiza la interconexi�n de estas clases.
	\item <<uso>> Utiliza una instancia de Dispositivo\_entrada\_caracteres para leer el fichero.
	\end{itemize}
\item \textbf{M�todos:}
	\begin{itemize}
	\item +cargar(d : Dispositivo\_entrada\_caracteres) : malla \newline
		Llama a cargar\_malla\_simple. Devuelve el resultado como malla, y no malla\_simple, con el fin de utilizarlo para alg�n tipo de cargador abstracto.
	\item +cargar\_malla\_simple(d : Dispositivo\_entrada\_caracteres) : Malla\_simple \newline
		Carga e instancia una Malla\_simple desde el dispositivo de entrada indicado.
	\end{itemize}
\end{itemize}

\subsection{El lexer}
El lexer se encarga de leer los caracteres del fichero y transformarlos en elementos de prean�lisis para el parser. A continuaci�n se describen estos elementos.

\newcommand{\elemento}[2]{\item \textbf{#1} \newline #2}
\begin{itemize}
\elemento{\emph{espacio en blanco}}{Ignorado}
\elemento{\emph{tabulador}}{Ignorado}
\elemento{\emph{retorno de carro}}{Ignorado}
\elemento{\#, \emph{cualquier caracter excepto retorno de carro}, \emph{retorno de carro}}{Ignorado, es un comentario}
\elemento{\emph{n�mero real}}{REAL}
\elemento{\emph{n�mero entero}}{ENTERO}
\elemento{", \emph{cadena de caracteres}, "}{CADENA}
\elemento{version}{VERSION}
\elemento{o}{OBJETO}
\elemento{nv}{NUMVERTICES}
\elemento{v}{VERTICE}
\elemento{nm}{NUMMATERIALES}
\elemento{m}{MATERIAL}
\elemento{tex}{TEXTURA}
\elemento{notex}{NOTEXTURA}
\elemento{esp}{ESPECULAR}
\elemento{emi}{EMISION}
\elemento{nf}{NUMCARAS}
\elemento{f}{CARA}
\elemento{fm}{MATERIAL\_CARA}
\elemento{fv}{VERTICE\_CARA}
\elemento{fvuc}{VERTICE\_CARA\_UV\_COLOR}
\elemento{\emph{fin de fichero}}{EOF}
\end{itemize}

\subsection{El parser}
El parser se encarga de procesar los elementos de prean�lisis del lexer, comprobar que siguen la sintaxis especificada y construir la malla a partir de la informaci�n del fichero. A continuaci�n se especifican las reglas del parser.

En may�sculas, los terminales definidos en la secci�n anterior. En min�sculas, los auxiliares. Cada palabra es un \emph{s�mbolo}. 

S�mbolo inicial: main\newline
\newcommand{\regla}[2]{\item\textbf{#1} -\textgreater \newline #2}
\begin{itemize}
\regla{main}{version nombre vertices materiales caras EOF}
\regla{version}{VERSION REAL}
\regla{nombre}{OBJETO CADENA}
\regla{vertices}{NUMVERTICES ENTERO desc\_vertices}
\regla{desc\_vertices}{desc\_vertice desc\_vertices \newline \textbar desc\_vertice}
\regla{desc\_vertice}{VERTICE REAL REAL REAL REAL REAL REAL}
\regla{materiales}{NUMMATERIALES ENTERO desc\_materiales}
\regla{desc\_materiales}{desc\_material desc\_materiales \newline \textbar desc\_material}
\regla{desc\_material}{MATERIAL CADENA textura difuso especular emision}
\regla{textura}{TEXTURA CADENA \newline \textbar NOTEXTURA}
\regla{difuso}{DIFUSO REAL REAL REAL REAL}
\regla{especular}{ESPECULAR REAL REAL REAL REAL REAL}
\regla{emisi�n}{EMISION REAL REAL REAL REAL}
\regla{caras}{NUMCARAS ENTERO desc\_caras}
\regla{desc\_caras}{desc\_cara desc\_caras \newline \textbar desc\_cara}
\regla{desc\_cara}{CARA material\_cara vertices\_cara}
\regla{material\_cara}{MATERIAL\_CARA ENTERO}
\regla{vertices\_cara}{vertice\_cara vertice\_cara vertice\_cara \newline \textbar vertice\_cara vertice\_cara vertice\_cara vertice\_cara}
\regla{vertice\_cara}{VERTICE\_CARA ENTERO \newline \textbar VERTICE\_CARA\_UV\_COLOR ENTERO REAL REAL REAL REAL REAL REAL}
\end{itemize}



