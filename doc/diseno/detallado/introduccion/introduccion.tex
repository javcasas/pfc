\chapter{Introducci�n}
Hasta ahora se ha especificado en general c�mo debe ser la forma y el comportamiento de las clases y del sistema. En este punto se discutir� al detalle c�mo debe ser implementado el sistema. Por ello, hasta este punto se ha estado hablando en t�rminos abstractos, en el c�mo deber�a funcionar. En este punto se especificar�n los mecanismos y detalles concretos que llevar� la implementaci�n.

\section{Restricciones y conversiones}
Para implementar este proyecto se utilizar� el lenguaje OCaML, que al igual que cualquier otro lenguaje de programaci�n tiene sus particularidades. Por ello, a continuaci�n se especifica una peque�a gu�a para transformar los modelos de an�lisis y dise�o a este lenguaje.

\begin{itemize}
\item \textbf{Tipos de dato}
	\begin{itemize}
	\item Void o vac�o, el tipo de dato que especifica que no se pasa nada de par�metro o no se devuelve nada, pasa a llamarse \emph{unit}.
	\item Conjunto ordenado, el tipo de dato que especifica una relaci�n con varias unidades siguiendo un determinado orden, pasa a ser \emph{array} o \emph{list}, dependiendo si es necesario el acceso r�pido a un elemento concreto o la capacidad de recorrer todos los elementos en orden.
	\item Conjunto, el tipo de dato que especifica una relaci�n con varias unidades, pasa a llamarse \emph{list}, por la facilidad de a�adir o filtrar elementos.
	\item Vector, Vector4, Vector3 y Vector2 son la misma clase con distintos nombres. La �nica diferencia reside en que si a Vector4, Vector3 o Vector2 se le pasa una cantidad de par�metros distinto al n�mero de la clase, se quejan y fallan con una excepci�n. Esto se hace con el fin de favorecer los controles de precondici�n y postcondici�n.
	\end{itemize}
\item \textbf{Operaciones del lenguaje}
	\begin{itemize}
	\item OCaML es un lenguaje fuertemente tipado, con inferencia de tipos. Debido a estas caracter�sticas, no es posible hacer \emph{downcasting}, es decir, transformar una referencia a una clase en una referencia a una de sus subclases. Esto se considera una operaci�n peligrosa y ambigua, y esa es la raz�n de que el lenguaje se niegue a implementarla.
	\item Debido a que el \emph{downcasting} es imposible, el \emph{upcasting} es una operaci�n de una s�la direcci�n, y no se puede invertir. Hay que crear caminos alternativos donde el \emph{downcasting} ser�a necesario.
	\item OCaML pone muchas trabas a la hora de crear dependencias circulares. Esto ha favorecido la creaci�n de un modelo de an�lisis y dise�o libre de dependencias circulares.
	\end{itemize}
\end{itemize}
