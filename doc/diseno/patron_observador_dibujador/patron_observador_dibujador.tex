\chapter{Patr�n Observador-Dibujador}
\section{Introducci�n}
El patr�n Observador-Dibujador define una dependencia entre un objeto conceptual y una forma o varias de dibujar dicho objeto, de tal manera que el mecanismo de dibujado del objeto est� desacoplado del objeto, y pueda cambiar sin que el objeto tenga que enterarse.
\subsection{Motivo}
El patr�n Observador-Dibujador tiene dos partes que son el Objeto y el Dibujador. Uniendo estas dos partes hay una mara�a de clases que a�aden funcionalidad para poder cambiar el Dibujador din�micamente.

Un ejemplo de uso de este patr�n es en el mecanismo de dibujado de una malla poligonal en pantalla, de tal manera que queda separado el modelo conceptual de malla del mecanismo que se utiliza para dibujar dicha malla en pantalla.
\subsection{Aplicabilidad}
Se puede usar el patr�n Observador-Dibujador en las siguientes situaciones:
\begin{itemize}
\item Cuando en un patr�n observador el sujeto desconoce c�mo debe ser tratado, o si su manera de ser tratado cambiar� en cualquier momento.
\end{itemize}
\subsection{Consecuencias}
\begin{itemize}
\item La fuerte separaci�n entre el sujeto y el observador-dibujador garantiza que es f�cil reaprovechar el sujeto.
\item Es sencillo cambiar radicalmente de mecanismo de dibujado cambiando din�micamente el observador-dibujador.
\end{itemize}
\subsection{Usos conocidos}
\subsection{Patrones relacionados}
\section{Estructura de clases}
\subsection{Colaboraciones}
\subsection{Detalles de implementaci�n}
\section{Descripci�n de las clases}
\section{Ejemplo de uso}
\section{Referencias}
