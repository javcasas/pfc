%\subsubsection{Relaciones}
%\begin{itemize}
%\end{itemize}
%\subsubsection{Atributos}
%\begin{itemize}
%\end{itemize}
%\subsubsection{M�todos}
%\begin{itemize}
%\item +inicializador \newline
%\end{itemize}
%\subsubsection{Par�metros de instanciaci�n}
%\begin{itemize}
%\end{itemize}

\chapter{Mecanismos de control general}
\section{Bucle principal}
\subsection{main\_loop}
\subsubsection{Relaciones}
\begin{itemize}
\item Depende de \Clase{Procesos.sistema\_impl}.
\item Depende de \Clase{Vista.motor}.
\end{itemize}
\subsubsection{Atributos}
\begin{itemize}
\item sistema\_procesos : Procesos.sistema \newline
El sistema de procesos que ejecutar� el bucle principal.
\item seguir\_bucle : boolean \newline
Guarda para saber cu�ndo debe terminar la ejecuci�n del bucle principal.
\end{itemize}
\subsubsection{M�todos}
\begin{itemize}
\item +sistema\_procesos : Procesos.sistema \newline
Getter.
\item +fin\_bucle \newline
Ordena la detenci�n del bucle principal al finalizar la vuelta actual.
\item +leer\_entrada \newline
Lee y procesa los eventos de entrada.
\item +ejecutar\_procesos \newline
Lanza la ejecuci�n del sistema de procesos.
\item +ejecutar\_motor\_grafico \newline
Lanza una actualizaci�n del motor gr�fico.
\item +ejecutar\_motor\_sonido \newline
Lanza una actualizaci�n del motor de sonido. Como actualmente no hay ning�n motor de sonido implementado, este m�todo no hace nada.
\item +mostrar\_resultados \newline
Lanza la ejecuci�n del motor gr�fico y de sonido.
\item +ejecutar \newline
Bucle principal. Mientras no se llame a fin\_bucle, llama en un bucle infinito a leer\_entrada, ejecutar\_procesos y mostrar\_resultados.
\end{itemize}

\subsection{main}
\subsubsection{Relaciones}
\begin{itemize}
\item Depende de \Clase{main\_loop}.
\item Depende de \Clase{Vista.motor}.
\item Depende de \Clase{Vista.proxy\_motor}.
\item Depende de \Clase{Procesos.sistema}.
\item Depende de \Clase{Logica.proceso\_base}.
\item Depende del subsistema SDL.
\end{itemize}
\subsubsection{M�todos}
\begin{itemize}
\item +inicializar\_SDL \newline
Inicializa el subsistema SDL, y establece el t�tulo de la ventana.
\item +detener\_SDL \newline
Detiene la biblioteca SDL.
\item +inicializar\_motor\_grafico \newline
Establece e inicializa el motor gr�fico "OpenGL".
\item +detener\_motor\_grafico \newline
Termina la ejecuci�n del motor gr�fico que est� instalado.
\item +inicializar\_motor\_sonido \newline
Establece e inicializa el motor de sonido. Como no hay ning�n motor de sonido implementado, no hace nada.
\item +detener\_motor\_sonido \newline
Termina la ejecuci�n del motor de sonido. Como no hay ning�n motor de sonido implementado, no hace nada.
\item +lanzar\_proceso\_base \newline
Instancia la clase \Clase{L�gica.proceso\_base} y lo a�ade al sistema de procesos.
\item +ejecutar \newline
Inicializa SDL, los motores gr�ficos y de sonido, lanza el proceso base y llama a ejecutar, del bucle principal. Cuando acabe, detiene los motores y la biblioteca SDL. Este m�todo es el punto de entrada del programa.
\end{itemize}


\section{L�gica}
\subsection{proceso\_base}
\subsubsection{Relaciones}
\begin{itemize}
\item Es una especificaci�n de \Clase{Procesos.proceso\_abstract}.
\end{itemize}
\subsubsection{M�todos}
\begin{itemize}
\item +ejecutar \newline
El punto de entrada de la l�gica de la aplicaci�n.
\end{itemize}
