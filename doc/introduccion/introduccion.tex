\chapter{Introducci�n}
El videojuego es uno de los �ltimos tipos de entretenimiento inventados. Heredando su legado de la imaginaci�n y la inform�tica, el videojuego es la especificaci�n en forma de programa inform�tico de lo que la mente del dise�ador pens� que ser�a divertido y entretenido.

\chapter{Propiedades del videojuego}
El videojuego tiene muchas propiedades, algunas heredadas de la inform�tica o el juego, y otras que aparecen en el desarrollo del mismo.
\section{Propiedades objetivas}
Con este nombre calificaremos las propiedades que son f�cilmente medibles.
\begin{itemize}
\item Coste de desarrollo \newline El videojuego, como programa inform�tico, tiene un coste que es lo que se gasta una empresa para desarrollarlo.
\item Calidad gr�fica \newline Define la calidad que tiene la parte visual del juego.
\item Calidad de sonido \newline Define la calidad que tiene la parte auditiva del juego.
\item Linealidad \newline Define la libertad que tiene el jugador dentro del juego para tomar decisiones.
\item Trama \newline Al igual que una pel�cula, la trama de un videojuego define una historia en la que el jugador participa, t�picamente como protagonista.
\item G�nero \newline El g�nero de un videojuego define el estilo de dificultades e interactividad.
\end{itemize}
\section{Propiedades subjetivas}
Con este nombre calificaremos las propiedades cuya medida sea dif�cil o imposible.
\begin{itemize}
\item Dificultad \newline
\item Adictividad \newline
\item Jugabilidad \newline
\item Diversi�n \newline
\item Inmersi�n \newline
\item Interactividad \newline

\end{itemize}
\chapter{G�neros del videojuego}
Al igual que hay distintos g�neros de pel�culas, pongamos por ejemplo, comedias, pel�culas b�licas, de esp�as o del oeste; tambi�n hay g�neros en el videojuego. El g�nero determina la interactividad y, a menudo, el argumento. A continuaci�n hay una lista de los g�neros m�s famosos del videojuego.
\begin{itemize}
\item FPS o First Person Shooters \newline
Juegos de disparos, donde el punto de vista es el que tendr�a el protagonista. El jugador juega dentro del papel de una persona envuelta en un mundo de armas de fuego en el que tiene que sobrevivir. Las capacidades necesarias en este tipo de juego suelen ser la capacidad de esconderse, la emboscada y la punter�a con las armas.
\item Estrategia \newline
Juegos en los que el jugador ve la acci�n pero no est� implicado directamente. T�picamente el jugador ve el mundo desde un punto de vista superior, de la misma manera que el jugador de ajedrez ve el tablero desde arriba. En estos juegos, el jugador env�a �rdenes a sus piezas, las unidades, que son las que deber�n luchar contra otras piezas para ganar la partida. Las capacidades necesarias en este tipo de juego suelen ser la organizaci�n, la planificaci�n y el conocimiento del enemigo.
\item Simuladores \newline
Juegos que simulan de una manera lo m�s realista posible situaciones del mundo real, o situaciones que podr�an darse en el mundo real. El jugador se ve envuelto en un entorno muy parecido a la realidad, en el que su habilidad determina el desenlace de la situaci�n. A menudo se simula el control de un veh�culo, t�picamente un avi�n. Las capacidades necesarias en este tipo de juego suelen ser las necesarias para la situaci�n real correspondiente.
\item Puzzles \newline

\end{itemize}
